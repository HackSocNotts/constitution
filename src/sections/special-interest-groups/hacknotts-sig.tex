\begin{clause}
    The \textbf{HackNotts Special Interest Group (SIG)} shall be responsible for the planning and running of the societies annual hackathon.
\end{clause}

\begin{subclause}
    The members of the HackNotts SIG shall be:
    \begin{itemize}[label=--,topsep=0em,itemsep=0em]
        \item Society President or Vice President
        \item HackNotts Organiser - Head of Finance
        \item HackNotts Organiser - Head of Logistices
        \item HackNotts Organiser - Head of Experience
        \item HackNotts Organiser - Head of Human Resources
    \end{itemize}
\end{subclause}

\begin{subclause}
    The HackNotts SIG shall appoint a member to be responsible to reporting the activites of the group.
\end{subclause}