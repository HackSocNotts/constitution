\begin{clause}
    Working groups serve as specialised teams within HackSoc focusing on a specific subculture, project, or expertise. Working groups aim to:
    \begin{enumerate}
        \item Encourage Collaboration: Working groups bring together individuals with a shared passion and expertise to collaborate on projects and initiatives that align with HackSoc's objectives.
        \item Promote skill or knowledge development: Working groups provide members with opportunities to enhance their technical, leadership, and interpersonal skills through hands-on projects, workshops, and knowledge sharing.
        \item Drive innovation: Working groups foster a culture of innovation by exploring new technologies, ideas, and approaches to problem-solving.
        \item Enrich the Community: Working groups contribute to the overall growth and vibrancy of HackSoc by organizing events, hosting workshops, and engaging with the broader community.
    \end{enumerate}
\end{clause}

\begin{subclause}
    Working Groups are mostly the same accross the board , but come in two main flavours:
    \begin{enumerate}
        \item Project Based: These working groups work and maintain some project defined in their charter.
        \item Topic Based: These working groups meet regularly to discuss or educate each other about different aspects of a certain subculture or expertise
    \end{enumerate}
\end{subclause}

\begin{clause}
    Establishing a Working Group consists of the following steps:
    \begin{itemize}
        \item Proposal: Any HackSoc member can propose the creation of a new working group by submitting a detailed proposal outlining the group's objectives, activities, and expected outcomes.
        \item Contestment: The proposal is passed to all of the existing sub-group officers. They can raise dispute with overlap with the objectives of newly proposed groups
        \item Approval: The proposal will be reviewed by the HackSoc leadership team, who will assess its alignment with HackSoc's mission and community needs. Final approval is decided by the president. If approved, the working group will be issued a "Hack Charter" and formally established. A member (normally the proposee) will be issued the title and privileges associated with being the Officer for that Working Group.
        \item Membership: Working groups are open to all HackSoc members interested in the group's focus area. Members can join or leave working groups based on their interests and availability.
    \end{itemize}
\end{clause}

\begin{clause}
    Working groups are headed by a Working Group Officer (Termed here as Officer; Usually called "`\$WORKING_GROUP_NAME` Officer". Other names may be used as long as it does not contain the words "President", "Secretary", or any other name that alludes to a committee position to avoid confusion with elected roles, e.g. Lead Justice, Head Developer, Project Lead, Hackiest Hacker, Brucest Bernard). Officers are put in place by and report to the HackSoc leadership team. The HackSoc leadership team reserves the ability to appoint the Officer and their successor, but the opinions of the Working Group's leadership and members should be taken into account wherever possible.
\end{clause}

\begin{subclause}
    Officers have the following responsibilities:
    \begin{enumerate}
        \item Coordinate and facilitate the group's activities, meetings, and projects.
        \item Define and communicate the group's objectives, goals, and timelines.
        \item Foster a collaborative and inclusive environment within the group.
        \item Act as a liaison between the working group and the HackSoc leadership team.
        \item Encourage skill development and mentorship opportunities within the group.
    \end{enumerate}
\end{subclause}

\begin{subclause}
    The HackSoc Leadership team reserve the right to remove and replace officers within reason (such as for prolonged abscence, missuse of society resource, breaching the HackSoc Code of Conduct, etc.).
\end{subclause}

\begin{subclause}
Working groups are encouraged to have regular meetings, communicate with their members through appropriate channels, and collaborate using suitable tools. These meetings and communication channels should be accessible to Working Group members, HackSoc Members, and HackSoc Committee equally.
\end{subclause}

\begin{subclause}
Working Groups should define clear goals, scope, and milestones. They should also identify the necessary resources, timelines, and potential risks associated with the Working Group. Working groups should regularly track and update the progress of their projects.
\end{subclause}

\begin{subclause}
Working groups have limited access to society resources including funds, room booking, and trip planning, with prior approval from the HackSoc leadership team (most of these tasks require the HackSoc Leadership team to perform either way). In addition, the society may choose to promote working group events in main channels or in newsletters and other communiques.
\end{subclause}

\begin{subclause}
Working Group Officers and their members are advised to adhere to HackSoc's code of conduct, with emphasis placed on the importance of creating a respectful and inclusive environment.
\end{subclause}

\begin{clause}
Working groups may undergo disbandment due to various reasons, such as achieving their objectives, lack of member engagement, or changes in the HackSoc community's needs. The disbandment process should be carried out in an organized and transparent manner.
\end{clause}

\begin{subclause}
The working group leadership and members, in collaboration with the HackSoc leadership team, should evaluate the continued relevance and effectiveness of the working group. This evaluation should consider factors such as the achievement of goals, member participation, and the ongoing impact on the HackSoc community.
\end{subclause}

\begin{subclause}
If the decision to disband a working group is being considered, it is essential to consult the working group members and obtain their input and feedback. This ensures that all perspectives are considered and that the decision is made collectively.
\end{subclause}

\begin{subclause}
Once the decision to disband a working group has been made, it should be communicated to the working group members and the wider HackSoc community. Transparent communication should explain the reasons for disbandment and express gratitude for the contributions made by the working group members.
\end{subclause}

\begin{subclause}
In the event of a working group's disbandment, the officer roles associated with the working group are also to be disestablished. This process ensures clarity and prevents confusion among members and the HackSoc leadership team. The working group officer should facilitate a smooth transition process. They should ensure that all relevant documentation, project files, and resources are properly transferred or archived for future reference.
\end{subclause}

\begin{subclause}
The officer role associated with the disbanded working group should be revoked, and any privileges granted to the officer should be removed. This includes access to communication channels, decision-making forums, and administrative controls related to the working group.
\end{subclause}

\begin{subclause}
In the event of an officer's unexpected and prolonged absence or disappearance without communication, the HackSoc leadership team will take immediate action to address the situation. The HackSoc leadership team will reach out to the officer through various communication channels to determine the reason for their absence and to offer support or assistance if needed. If the officer remains unreachable for an extended period, the HackSoc leadership team, in consultation with the working group members, will assess the impact of the officer's absence on the group's operations and the community, and if necessary, the HackSoc leadership team will appoint an interim officer or redistribute responsibilities within the working group to ensure its continued functioning. Efforts will be made to maintain the integrity of the working group's projects, activities, and resources during the officer's absence. Once the officer resumes communication or returns, a thorough discussion and evaluation will take place to decide the best course of action for the working group moving forward. The committee may decide to appoint a new officer if need be.
\end{subclause}